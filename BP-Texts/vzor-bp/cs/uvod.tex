\chapter*{Úvod}
\addcontentsline{toc}{chapter}{Úvod}

Na internete existujú stránky, ktoré ponúkajú funkcionalitu spojenú s mapami.
Jedná sa o klasické pozeranie máp, vyhľadávanie objektov na mape, alebo znázornenie  
dát pomocou mapy. V súčasnej dobe na internete neexistuje webová aplikácia pre cestovateľov alebo aj bežných ľudí, ktorá 
by vhodne umožňovala spravovanie objektov na mape.
Pod pojmom "spravovanie" myslime nasledovné funkcionality:
\begin{itemize}
    \item Zaznamenávať si objekty na mape.
    \item Priradiť objekt do kolekcie objektov.
    \item K objektu priradiť poznámky a nastaviť objektu ikonu, ktorá reprezentuje objekt ako  
    bod na mape.
    \item Funkcionalita návštevnosti, teda možnosť zaznamenať, či daný objekt bol 
    používateľom navštívení a poprípade poznamenať aj dátum návštevy.
\end{itemize}

Ďalšia chýbajúca funkcionalita je detailnejšie vyhľadávanie objektov, ktoré splňujú požiadavky na základe parametrov 
zadaných používateľom ako: 
\begin{itemize}
    \item Definovať aké typy objektov majú byt vyhľadávane, typom myslime skupinu objektov, ktoré majú spoločné vlastností
    a človek ich zaradí do rovnakej skupiny. Typ môže byt napr. hrad, mesto, múzeum... 
    \item Vymedziť územie, kde hľadáme objekty.  
    \item  Definovať a zadať rôzne vlastnosti objektu, ktoré sú rozumne pre daný typ objekt.
    Napríklad pre objekt reprezentujúci mesto vieme vymedziť vo vyhľadávaný aby nájdene mesta 
    splňovali vlastnosť, a tou je vlastnosť populácia, čo reprezentuje číslo počtu obyvateľstva
    daného mesta aby bolo väčšie ako jeden milión.  
\end{itemize}

Posledná funkcionalita je možnosť poskytovania informácii o konkrétnom objekte. 

Webová služba Google maps( to do : odkaz) poskytuje možnosť spravovania objektov, ale bez rozumnej 
funkcionality nastaviť návštevnosť. Vieme zaznačiť objekt na mape ako bod, ktorému vieme 
nastaviť ikonu a farbu, napísať poznámky a priradiť do kolekcie objektov. Avšak nevieme rozumne zaznamenať,
či bol daný objekt navštívený. Museli by sme si to jedine poznamenať do poznámok a  
pre grafickú reprezentáciu by sme mohli napríklad používať sivú farbu ikonky pre zatiaľ nenavštívený objekt. 
Google maps taktiež poskytujú možnosť vyhľadávať konkrétne objekty podľa názvu objektu, alebo hľadaním 
v blízkosti objekty podľa zadaného kľúčového slova, ktoré opisuje nejakú skupinu objektov. Jedna sa o typ objektov,
ako sme už spomenuli. Nenájdeme tu možnosť zadefinovať lepšie naše požiadavky, ktoré by ovplyvňovali a 
vymedzovali vyhľadávanie. Možnosť poskytovania informácii o konkrétnom objekte Google maps neposkytujú. Informácie by sme museli 
vyhľadávať cez rôzne stránky, poskytujúce informácie. 

Touto pracou chceme vytvoriť 
webovú aplikáciu, ktorá by obsahovala uvedené funkcionality. Vďaka tejto aplikácii by sa uľahčilo spravovanie a vyhľadanie 
mapových objektov pre skupinu používateľov ako sú napríklad cestovatelia. 

Cieľom práce je:
\begin{itemize}
    \item analyzovať používateľské požiadavky
    \item analyzovať zdroje a získavanie dát, ktoré bude aplikácia využívať
    \item navrhnúť a implementovať webovú aplikáciu
    \item navrhnúť dátový model, ktorým budeme ukladať dáta do databázy a vybrať vhodnú databázu pre tento projekt
    \item aplikáciu otestovať
\end{itemize}

Práca opisuje základne technológie a princípy, ktoré sme v prací použili, analýzu používateľských požiadavkou, ktoré sú očakávané od aplikácie. Následne sú  
zanalyzovane dáta, ktoré potrebujeme a dotazy pre získavanie týchto vyhovujúcich dát na základe vstupných 
parametrov. Rozoberanie do detailov aké sú možnosti definovať obmedzenia vyhľadávania objektov a ako museli byt optimalizované aby sme dostali výsledky v rozumnom čase. 

Ďalej popíšeme navrch architektúry aplikácie. 

V implementačnej časti je opísaný vývoj aplikácie. Aké technológie boli použité, podrobne si popíšeme 
jednotlivé časti programu a ako medzi sebou tieto časti interagujú. 

V užívateľskej časti je opísaný jednoduchý postup ako pracovať s aplikáciou, spravovať objekty a hlavne 
ako vyhľadávať objekty na základe parametrov. Nájdeme tam aj konkrétne príklady, ktoré ukazujú konkrétne vyhľadávania s danými parametrami. 

Následne v poslednej časti ukážeme ako sme aplikáciu otestovali. 
