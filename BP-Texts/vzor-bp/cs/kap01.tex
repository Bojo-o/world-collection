%%% Fiktivní kapitola s ukázkami sazby

\chapter{Analýza požiadavkou}

V tejto kapitole sú najprv uvedené jednoduché definície požiadavkou ktoré sa
zameriavajú na funkcionalitu aplikácie. Uvedieme skupiny do ktorých sa požiadavky delia na základe zámeru požiadavkou.
Následne sú analyzované funkčné a kvalitatívne (nefunkčne) požiadavky očakávane od aplikácie.
Jednotlivé požiadavky sú spísane v tejto kapitole.
Na konci kapitoly sú uvedené aj požiadavky ktoré boli počas návrhu a implementácie zamietnuté.

\section{Požiadavky}
Pre navrhnutie a vytvorenie úspešného softwarového projektu je potrebné na začiatku identifikovať a
definovať rozumné vlastnosti, ktoré stakeholderi požadujú a očakávajú od aplikácie. V rámci tejto práce je stakeholder
používateľ, ktorý bude používať našu aplikáciu. Používateľom môže byť hoci aký človek, ale zámer našej aplikácie prevažne
mieri na užšiu doménu používateľov. Do tejto domény patria cestovatelia, turisti a návštevníci kultúrnych miest.

Spomínane vlastnosti je potrebné vyjadriť a opísať v štruktúrovanej forme.
Požiadavka opisuje a definuje vlastnosť, ktorú používateľ vyžaduje od aplikácie.
Vďaka požiadavkám je jednoduchšie navrhnúť architektúru aplikácie, implementovať aplikáciu a následne
nám to pomôže aj aplikáciu otestovať.

Požiadavky sa delia do dvoch hlavných skupín. Jednou skupinou sú takzvané funkčné požiadavky, druhou
sú kvalitatívne požiadavky. Kvalitatívne požiadavky sa môžu označovať aj ako nefunkčné požiadavky.

Funkčne požiadavky sa dajú jednoducho definovať ako služby, ktoré sú poskytované aplikáciou. Inak povedané čo všetko
dokáže aplikácia urobiť, akú funkcionalitu je schopná poskytnúť používateľovi. Ďalej definuje kto bude využívať konkrétne služby.
Funkčné požiadavky zároveň
opisujú vstupné dáta a snažia sa prezentovať očakávaný výstup. Ak aplikácia nespĺňa niektorý z
funkčných požiadavkou, potom je možne aplikáciu prehlásiť za nefunkčnú.

Kvalitatívne požiadavky alebo aj nefunkčné požiadavky sa dajú jednoducho definovať ako požiadavky,
ktoré opisujú ako dobre aplikácia poskytuje svoje služby. Zameriavajú sa na to ako systém zvládne
splniť funkčnú požiadavku. Na tieto požiadavky je potrebne sa pozerať z pohľadu rôznych kvalít ako sú
napríklad výkon a dostupnosť.


\section{Funkčné požiadavky}
Na základe úrovni poskytnutých detailov sa funkčné požiadavky delia na :
\begin{enumerate}
      \item Používateľské požiadavky. Tie sú spísane pre používateľov aplikácie v prirodzenom
            jazyku, ktorý je zrozumiteľný aj bez technických znalostí. Na vyjadrenie používateľských požiadavkou sa používajú
            takzvané používateľské príbehy.
      \item Systémové požiadavky. Tie sú na rozdiel od používateľských požiadavkách štruktúrované vo forme
            scenárov. Na vyjadrenie sa používajú takzvané use cases(prípady použitia), kde sú detaily opísané do väčšej hĺbky.
            Zároveň definujú čo by malo byť implementované.
\end{enumerate}

Na začiatku prace prebehla analýza funkčných požiadavkou, ktoré by používateľ mohol očakávať od aplikácie.

Počas analýzy a samotnej implementácie aplikácie sme niektoré funkčné požiadavky pridali a tým obohatili funkcionalitu aplikácie
na základe získania nových vedomosti . Na druhu stranu boli niektoré zamietnuté z dôvodu časovej
tiesne alebo usúdenia, že tieto funkčné požiadavky sa nedajú rozumne implementovať.


\subsection{Funkčne požiadavky ako používateľské príbehy}
V tejto sekcii je výpis všetkých funkčných požiadavkou vyjadrených formou používateľských príbehov.

Používateľský príbeh je vysvetlenie funkcie aplikácie neformálnou formou z pohľadu
koncového používateľa aplikácie. Cieľom používateľského príbehu je jednouchého, bez
podrobnejších detailov, opísať funkčne požiadavky. Každý používateľsky príbeh má nasledujúcu formu:

\begin{code}
      Ako <druh používateľa>,
      som schopný/potrebujem <urobiť niečo>,
      pretože <dôvod>.
\end{code}

Používateľské príbehy sú súčasťou agilného vývoja,
pretože dôraz kladú na potreby používateľa, ktoré sa môžu počas vývoja meniť.

Nasleduje výpis jednotlivých funkčných požiadavkou, ktoré sú rozdelene do dvoch skupín na základe funkcionality.
Prvou skupinou sú funkčne požiadavky pre vyhľadávanie a získavanie dát a tou druhou sú funkčné požiadavky spravujúce mapové
objekty.

Ešte predtým než uvedieme funkčne požiadavky si vysvetlime zopár pojmov, ktoré sa
v nich vyskytujú.

\begin{itemize}
      \item Pojmom "objekt" je myslený mapový objekt, ktorý je možne reprezentovať ako bod na mape.
      \item Pojmom "kolekcia" je myslený zoznam, list , do ktorého sa priradzujú, ukladajú objekty. Každá kolekcia
            má svoje jedinečne meno. Kolekciu si vytvára sám používateľ aplikácie a je na ňom aké objekty do kolekcie budú patriť.
      \item Pojmom "zoznam tried, ktorých je objekt inštanciou" rozumieme zoznam všetkých tried, respektíve kategórii ktoré definujú konkrétnejšie objekt. Napríklad pre objekt mesto “Praha” tento zoznam
            obsahuje triedy ako “veľké mesto”, “hlavné mesto” a “turistická destinácia”. Tento zoznam poskytuje používateľovi informáciu čo je daný objekt zač, pod akú kategóriu spadá.
      \item Pojmom "detaily" objektu rozumieme zoznam dvojíc, kde prvý člen je kľuč definujúci názov vlastnosti objektu a druhý člen je hodnota tejto vlastnosti.
            Napríklad dvojica "kontinent" a "Európa" nás informuje o tom, že objekt je súčasťou kontinentu Európa.
\end{itemize}

Funkčne požiadavky pre vyhľadávanie a získavanie dát:
\begin{enumerate}
      \item Používateľ je schopný vyhľadať konkrétny mapový objekt podľa zadania názvu objektu, pretože si ho môže chcieť uložiť do svojej kolekcie.
      \item Používateľ potrebuje od aplikácie aby mu pre vyhľadaný objekt získala informácie definujúce objekt ako súradnice, meno, popis a zoznam tried, ktorých je objekt
            inštanciou, aby mal používateľ o objekte informácie identifikujúce objekt.
      \item Používateľ potrebuje od aplikácie aby mu pre vybraný objekt získala obrázok objektu, aby používateľ mal predstavu ako objekt vyzerá.
      \item Používateľ je schopný požiadať aplikáciu aby mu pre vybraný objekt vyhľadala odkaz na článok, sídliaci na Wikipédii, ak článok existuje, aby používateľ mal možnosť
            sa o objekte dozvedieť viac informácii vo forme článku.
      \item Používateľ je schopný požiadať aplikáciu aby mu získala podrobnejšie informácie o objekte(budeme označovať ako detaily objektu), medzi ktorými vie používateľ
            filtrovať podľa názvu informácie, aby používateľ získal konkrétnu informáciu o objekte.
      \item Používateľ je schopný vyhľadať skupinu objektov, ktoré spĺňajú všetky parametre zadané používateľom, aby si používateľ mohol vyhľadať  skupinu objektov na základe
            svojich preferencii, ktoré si môže následne uložiť do svojej kolekcie.
      \item Používateľ je schopný v procese vyhľadávania objektov zadať parameter definujúci triedu, kategóriu, ktorá obmedzí výsledky vyhľadávania, pretože používateľ chce vyhľadávať
            objekty, ktoré definuje táto trieda. Triedou môže byt napríklad hrad, mesto alebo múzeum. Trieda definuje, že vyhľadané objektu musia byť hrady, mesta alebo múzea.
      \item Používateľ potrebuje od aplikácie možnosť vyhľadať možne a rozumné triedy definujúce objekty podlá názvu triedy, aby sa následne trieda dala použiť ako parameter.
      \item Používateľ je schopný v procese vyhľadávania objektov definovať respektíve vymedziť územie, ktoré obmedzí vyhľadávanie, pretože používateľ chce vyhľadávať objekty nachádzajúce sa
            na konkrétnom území.
      \item Používateľ je schopný vyhľadať podlá názvu a definovať parameter územia ako súčasne existujúce územie, aby používateľ vedel vyhľadávať objekty
            na území kontinentu, krajiny alebo administratívneho územia.
      \item Používateľ je schopný definovať parameter územia ako kruhovú oblasť, ktorú používateľ definuje nastavením súradníc bodu stredu kruhu a nastavaním polomeru kruhu, aby vedel používateľ
            vyhľadávať objekty v okolí zvoleného miesta.
      \item Pre spomínanú definíciu kruhu, v ktorom sa budú objekty vyhľadávať, používateľ je schopný stred tohto kruhu nastaviť na mape posúvaním bodu, alebo
            vyhľadaním konkrétneho miesta pomocou vyhľadávania. V takom prípade sa stred kruhu presunie na súradnice nájdeného a vybratého objektu. Pretože používateľ
            môže chcieť vyhľadávať objekty v blízkom okolí okolo konkrétneho miesta.
      \item Používateľ je schopný v procese vyhľadávania objektov definovať nejaké vlastnosti objektu, ktoré musia nájdene objekty splňovať, aby používateľ mohol zadať vlastnosti,
            ktoré očakáva, že nájdené objekty budú splňovať.
      \item Používateľ potrebuje od aplikácie aby mu poskytla zoznam možných vlastnosti, ktoré vie definovať v procese vyhľadávania objektov, aby používateľ vedel aké vlastnosti je schopný definovať.
      \item Používateľ potrebuje od aplikácie prostriedky pre zadávanie konkrétnych hodnôt vlastnosti, aby používateľ vedel tieto vlastnosti definovať.
\end{enumerate}


Funkčne požiadavky pre spravovanie mapových objektov:
\begin{enumerate}
      \item Používateľ je schopný vytvoriť prázdnu kolekciu neobsahujúcu na začiatku žiadne objekty, aby používateľ vedel do nej priradzovať objekty.
      \item Používateľ je schopný si nájdený objekt uložiť do niektorej z existujúcich kolekcii, pretože sa používateľ rozhodol tento objekt začleniť do svojej zbierky.
      \item Používateľ je schopný editovať názov kolekcie, pretože sa používateľ rozhodol zmeniť jej názov.
      \item Používateľ je schopný odstrániť kolekciu, ktorá pri zmazaní zmaže aj svoj obsah, teda všetky objekty, ktoré obsahovala, pretože používateľ sa rozhodol, že už ďalej nechce spravovať tuto kolekciu a objekty v nej.
      \item Používateľ je schopný editovať názov uloženého objektu. Názov nemusí byt unikátny.
      \item Používateľ je schopný odstrániť konkrétny objekt z kolekcie, pretože už ho nechce mať naďalej uložený.
      \item Používateľ je schopný napísať a uložiť pre každý uložený objekt textové poznámky, ktoré aplikácia umožni stále editovať, aby si používateľ mohol zaznačiť nejaké poznámky pre konkrétny objekt.
      \item Používateľ je schopný nastaviť ikonu objektu na mape pre každý uložený objekt, z pred definovateľného výberu možných obrázkov ikoniek, aby vedel požívateľ z podhľadu na mapu určiť kategóriu objektu na základe tejto ikonky.
      \item Používateľ potrebuje od aplikácii aby mu umožnila vidieť jeho zoznam kolekcii, aby ich vedel spravovať.
      \item Používateľ potrebuje od aplikácie aby pre vyberanú kolekciu sa na mape zobrazili body s ikonkou reprezentujúce všetky uložene objekty v tejto kolekcii, aby si vedel používateľ prezerať a spravovať svoje objekty.
      \item Používateľ potrebuje aby mu aplikácia umožnila nastaviť pre konkrétny uložený objekt informáciu o návštevnosti, aby si mohol používateľ označiť, že daný objekt už navštívil.
      \item Používateľ je schopný zadať pri nastavovaní návštevnosti aj dátum návštevy, ktorý môže byť vyjadrený ako konkrétny dátum, mesiac v roku alebo rok.
            Zároveň je používateľ schopný vyjadriť dátum návštevy ako rozsah časového obdobia, od kedy , do kedy a to ako dátumy alebo mesiace v roku.
            Pretože používateľ si chce zaznamenať aj dátum návštevy objektu alebo časové obdobie návštevy.
      \item Používateľ potrebuje aby mu aplikácia prezentovala informáciu o navštívení objektu, aby používateľ vedel či tento objekt už navštívil.
      \item Používateľ je schopný zlúčiť dve kolekcie do jednej, výberom jednej kolekcie, ktorá sa vymaže a jej obsah bude prenesený do druhej
            vybratej kolekcie, aby mohol požívateľ rýchlo zlúčiť obsah dvoch kolekcii.
      \item Používateľ je schopný nastaviť aby všetky uložené objekty v danej kolekcie mali rovnaký obrázok ikonky zobrazení na mape, pretože používateľ nechce pre každý objekt v kolekcii samostatne nastavovať ikonku,
            ak chce aby všetky objekty v danej kolekcii mali rovnakú ikonku.
      \item Používateľ môže kliknúť na niektorú ikonu reprezentujúcu uložený objekt. Po kliknutí sa objaví obdĺžnik s informáciami o danom objekte.
      \item Používateľ potrebuje od aplikácie aby mu v zozname kolekcii, prezentovala informáciu koľko každá kolekcia obsahuje celkovo objektov,
            a koľko z týchto objektov sú označene používateľom ako navštívene, aby používateľ mal informácie o svojich kolekciách.
      \item Používateľ je schopný nájdene objekty v procese vyhľadávania objektov uložiť do niektorej zo svojich kolekcii, aby používateľ následne nemusel samostatne vyhľadávať objekty, ktoré si chce uložiť do kolekcie.
      \item Používateľ potrebuje od aplikácie aby mu nájdene objekty v procese vyhľadávania objektov prezentovala buď v tabuľke alebo na mape, kde každý výsledok je zaznačený bodom na mape, aby mal používateľ prehlaď o nájdenej skupine objektov.
      \item Používateľ je schopný filtrovať podlá mena v skupinke nájdených objektov, pretože používateľ chce zistiť či sa v tejto skupine nachádza konkrétny objekt.
      \item Používateľ je schopný zmeniť meno objektu alebo niektorý nájdený objekt odstrániť zo zoznamu výsledkov pred uložením do kolekcie, pretože používať chce niektoré objekty pred uložením editovať.
      \item Používateľ potrebuje od aplikácie aby mu vedela vrátiť späť vykonanú akciu pri editovaní a mazaní nájdených výsledkov.
\end{enumerate}

\subsection{Funkčne požiadavky ako use case scenáre }
V tejto pod-sekcii sú vyjadrené funkčne požiadavky ako use cases scenáre.
Z dôvodu veľkého množstva funkčných požiadavkou sú ako use case scenáre popísane iba niektoré vybraté funkčne požiadavky,
pre ilustrovanie toho ako by vyzerala funkčná požiadavka vyjadrená ako use case scenár.

Use case scenár je štruktúrovaný opis interakcii, medzi hercami (actors) a systémom, definujúcich správanie. V našej práci hercom je používateľ aplikácie.
Každý use case reprezentuje správanie, ktoré môžeme špecifikovať ako use case scenár.

Tento štruktúrovaný opis ma nasledujúcu formu:
\begin{enumerate}
      \item situácia na začiatku - opis stavu systému v akom je pred začiatkom interakcii
      \item normálna interakcia – normálna interakcia medzi používateľom a systémom
      \item alternatívna interakcia - uvádzajú sa alternatívne scenáre ak sa niečo pokazilo počas normálneho interagovania
      \item stav systému po skončení - opis stavu systému v akom sa systém nachádza po interakcii
\end{enumerate}

Výpis niektorých vybratých funkčných požiadavkou ako use case scenáre:

\begin{itemize}
      \item Používateľ je schopný vyhľadať konkrétny mapový objekt podľa zadania názvu objektu,
            pretože si ho môže chcieť uložiť do svojej kolekcie.
            \begin{itemize}
                  \item Situácia na začiatku:
                        \begin{enumerate}
                              \item Požívateľ má otvorenú webovú stránku.
                        \end{enumerate}
                  \item Normálna interakcia:
                        \begin{enumerate}
                              \item Používateľ klikne v menu na tlačidlo "Add collectible".
                              \item Používateľ si otvorí bočné menu.
                              \item Používateľ vyplní časť alebo celý názov objektu do vyhľadávacieho poľa.
                              \item Používateľ si vyberie a klikne na objekt z ponúknutého výberu nájdených objektov.
                        \end{enumerate}
                  \item Alternatívna interakcia:
                        \begin{enumerate}
                              \item Zadaný názov neodpovedá žiadnemu objektu. V takomto prípade aplikácia neposkytne žiaden vyber objektov.
                              \item Služba poskytujúca toto vyhľadávanie z nejakého dôvodu nefunguje. V takom prípade aplikácia upozorni používateľa, že vyhľadávanie zlyhalo z dôvodu nefunkčnosti služby.
                        \end{enumerate}
                  \item Stav systému po skončení:
                        \begin{enumerate}
                              \item Vybratý objekt je zabránený na mape a sú k nemu poskytnuté informácie.
                              \item Vybratý objekt je pripravený na možnosť uloženia do niektorej z kolekcii.
                        \end{enumerate}
            \end{itemize}

      \item Používateľ je schopný požiadať aplikáciu aby mu pre vybraný objekt vyhľadala odkaz na článok, sídliaci na Wikipédii, ak článok existuje, aby používateľ mal možnosť sa o objekte dozvedieť viac informácii vo forme článku.
            \begin{itemize}
                  \item Situácia na začiatku:
                        \begin{enumerate}
                              \item Používateľ ma konkrétny objekt uložený v niektorej kolekcii.
                              \item Požívateľ má otvorenú webovú stránku.
                        \end{enumerate}
                  \item Normálna interakcia:
                        \begin{enumerate}
                              \item Používateľ klikne v menu na tlačidlo "Home" , alebo už je stránka v danom stave.
                              \item Používateľ si otvorí bočné menu.
                              \item Používateľ vyberie kolekciu zo zoznamu svojich kolekcii.
                              \item Aplikácia zobrazí objekty na mape.
                              \item Používateľ klikne na konkrétny objekt na mape.
                              \item Aplikácia zobrazí UI pre mapový objekt.
                              \item Používateľ klikne na tlačidlo "Show Details".
                              \item Aplikácia získa detaily o objekte, poslaním POST metódy s parametrami identifikujúce objekt na server.
                              \item Aplikácia zobrazí získané detaily, medzi ktorými je aj odkaz na článok sídliaci na Wikipédii.
                              \item Používateľ klikne na odkaz.
                        \end{enumerate}
                  \item Alternatívna interakcia:
                        \begin{enumerate}
                              \item Vybraný objekt nemá článok na Wikipédii. V takomto prípade sa v UI neobjaví odkaz na článok
                                    v zozname detailov.
                        \end{enumerate}
                  \item Stav systému po skončení:
                        \begin{enumerate}
                              \item Používateľ má otvorené nové okno kde je článok objektu.
                              \item Aplikácia ukazuje naďalej detaily objektu.
                        \end{enumerate}
            \end{itemize}

      \item Používateľ je schopný vytvoriť prázdnu kolekciu neobsahujúcu na začiatku žiadne objekty, aby používateľ vedel do nej priradzovať objekty.
            \begin{itemize}
                  \item Situácia na začiatku:
                        \begin{enumerate}
                              \item Používateľ ma otvorenú stránku, kde je aplikácia v stave ukladania objektov do kolekcie.
                              \item V aplikácii sú vybraté nejaké mapové objekty.
                        \end{enumerate}
                  \item Normálna interakcia:
                        \begin{enumerate}
                              \item Používateľ klikne na tlačidlo "Create a new Collection".
                              \item Aplikácia zobrazí nové UI.
                              \item Používateľ zadá do poľa názov kolekcie, ktorú chce vytvoriť.
                              \item Používateľ klikne na tlačidlo "Save"
                              \item Aplikácia pošle na server príkaz na vytvorenie novej kolekcie.
                        \end{enumerate}
                  \item Alternatívna interakcia:
                        \begin{enumerate}
                              \item Zadaný názov kolekcie ma menej ako tri znaky. V takomto prípade aplikácia
                                    neumožni používateľovi kliknúť na tlačidlo "Save" pre vytvorenie novej kolekcie a upozorni používateľa o probléme.
                              \item Zadaný názov už je zabratý niektorou existujúcou kolekciou. V takomto prípade aplikácia
                                    neumožni používateľovi kliknúť na tlačidlo "Save" pre vytvorenie novej kolekcie a upozorni používateľa o probléme.
                        \end{enumerate}
                  \item Stav systému po skončení:
                        \begin{enumerate}
                              \item V systéme je uložená nová prázdna kolekcia.
                        \end{enumerate}
            \end{itemize}

      \item Používateľ je schopný odstrániť kolekciu, ktorá pri zmazaní zmaže aj svoj obsah, teda všetky objekty, ktoré obsahovala, pretože používateľ sa rozhodol, že už ďalej nechce spravovať tuto kolekciu a objekty v nej.
            \begin{itemize}
                  \item Situácia na začiatku:
                        \begin{enumerate}
                              \item Používateľ ma otvorenú stránku.
                              \item V systéme je uložená aspoň jedna kolekcia.
                        \end{enumerate}
                  \item Normálna interakcia:
                        \begin{enumerate}
                              \item Používateľ klikne na tlačidlo "Edit Collections".
                              \item Aplikácia zobrazí editačnú tabuľku s kolekciami používateľa.
                              \item Používateľ nájde alebo vo vyhľadávacom poli zadá názov kolekcie a vyberie riadok s konkrétnou kolekciou.
                              \item Používateľ klikne na tlačidlo "Remove" v danom riadku.
                              \item Aplikácia pošle na server príkaz na vymazanie danej kolekcie.
                        \end{enumerate}
                  \item Alternatívna interakcia:
                        \begin{enumerate}
                              \item Server nevykoná príkaz, alebo nie je funkční. V takom prípade sa kolekcia nevymaže a používateľ ju má stále uloženú v systéme.
                        \end{enumerate}
                  \item Stav systému po skončení:
                        \begin{enumerate}
                              \item Kolekcia je úspešne vymazaná zo systému.
                              \item Všetky objekty, ktoré boli priradene k vymazanej kolekcii sú taktiež všetky úspešne vymazané zo systému.
                        \end{enumerate}
            \end{itemize}

      \item Používateľ potrebuje aby mu aplikácia umožnila nastaviť pre konkrétny uložený objekt informáciu o návštevnosti, aby si mohol používateľ označiť, že daný objekt už navštívil.
            \begin{itemize}
                  \item Situácia na začiatku:
                        \begin{enumerate}
                              \item V systéme je uložená aspoň jedna kolekcia s aspoň jedným objektom, ktorý do nej patri.
                              \item Používateľ ma otvorenú stránku.
                        \end{enumerate}
                  \item Normálna interakcia:
                        \begin{enumerate}
                              \item Používateľ klikne v menu na tlačidlo "Home".
                              \item Používateľ si otvorí bočné menu.
                              \item Používateľ vyberie kolekciu zo zoznamu svojich kolekcii.
                              \item Aplikácia zobrazí objekty na mape.
                              \item Používateľ klikne na konkrétny objekt na mape.
                              \item Aplikácia zobrazí UI pre mapový objekt.
                              \item Používateľ klikne na tlačidlo "Visitation"
                              \item Aplikácia zobrazí UI so zaškrtávacím okienkom na označenie návštevy.
                              \item Používateľ zaznačí zaškrtávacie okienko.
                              \item Používateľ klikne na tlačidlo "Save".
                              \item Aplikácia pošle na server príkaz potrebnými parametrami pre nastavenie návštevnosti objektu.
                              \item Aplikácia zobrazí používateľovi, že daný objekt bol navštívený.
                        \end{enumerate}
                  \item Alternatívna interakcia:
                        \begin{enumerate}
                              \item Server nevykoná príkaz, alebo je v nefunkčnom stave. V takom prípade používateľ uvidí v aplikácii, že daný objekt navštívil,
                                    ale v systéme je zaznačene, že objekt je ešte nenavštívený. Po znovu načítaný stránky používateľ uvidí, že objekt nie je navštívený.
                        \end{enumerate}
                  \item Stav systému po skončení:
                        \begin{enumerate}
                              \item Objekt ma v systéme uložene dáta o tom, že bol používateľom navštívený.
                        \end{enumerate}
            \end{itemize}
\end{itemize}



\subsection{Zamietnuté funkčne požiadavky }
Nasleduje výpis funkčných požiadavkou, ktoré boli zamietnuté na základe dvoch faktorov.

\begin{enumerate}
      \item Čas. Z dôvodu časovej tiesne boli niektoré požiadavky zamietnuté.
            Tieto požiadavky by ale mohli byť v budúcnosti implementovane.
      \item Druhy faktor je zamietnutie po analýze
            dátových zdrojov a možností, po ktorých sme došli k záveru, že ich nevieme rozumne
            implementovať. Tým nehovoríme, že by sa nedali implementovať.
\end{enumerate}

Zamietnuté funkčne požiadavky:

\begin{enumerate}
      \item Používateľ je schopný ako oblasť, v ktorej sa bude skupina objektov vyhľadávať, definovať aj nie súčasne existujúce územia.
            Jedna sa napríklad o štáty, ktoré v dnešnej dobe neexistujú.
      \item Aplikácia dôkaze mapový objekt reprezentovať aj ako územnú oblasť definovane množinou bodov na mape.
            Teda aplikácia dôkaze vykresliť na mape aj polygón definujúci oblasť objektu.
            Vhodné na zaznačovanie navštívených oblastí ako napríklad štáty sveta.
\end{enumerate}

\section{Nefunkčné požiadavky}
Nasledujúce požiadavky môžeme rozdeliť do kategórii podľa zamerania ako napríklad výkon aplikácie, spoľahlivosť dát alebo rozšíriteľnosť.
Týchto rôznych kategórii existuje obrovské množstvo a preto sú uvedene iba niektoré konkrétne, ktoré sa tykajú tejto aplikácie.

\begin{enumerate}
      \item Aplikácia je dostatočne vykoná. Vyhľadávanie dát na základe zadania názvu v niektorom z vyhľadávačov.
            netrvá viac ako sekundu.
      \item Aplikácia je spoľahlivá. Vyhľadávanie vo konkrétnych vyhľadávačov poskytuje rozumne výsledky, teda výsledky ktoré
            sú spojene s tým čo je vyhľadávane.
      \item Vyhladievanie skupiny objektov na zakladá vstupných parametrov skonči do minúty. Ak vyhľadávanie neskonči do tohto času, tak
            aplikácia vyhlási vyhľadávanie ako neúspešne z dôvodu zadania náročných a nevhodných parametrov pre vyhľadávanie.
      \item V prípade nefunkčnosti vyhľadávania aplikácia sa správa rozumne, teda poskytne informáciu používateľovi o naskytnutom probléme.
      \item Pri zmene dát uloženého mapového objektu, aplikácia poskytne používateľovi správu o tom, či zmena bola úspešne uložená do systému. Napríklad po editovaní poznámok alebo zadania návštevnosti.
      \item Pre pomenovanie názvu kolekcie, ktoré už iná kolekcia používa, aplikácia neumožni vytvoriť novu kolekciu a nahlási
            používateľovi informáciu o tom, že sa snaží použiť už zabratý názov.
      \item Aplikácia je navrhnutá tak aby ju používateľ intuitívne vedel ovládať.
      \item Aplikácia je implementovaná ako single-page webová stránka.
      \item Aplikácia je rozdelená na časti backend s databázou a frontend, ktoré medzi sebou komunikuje cez API.
      \item Frontend je implementovaný v súčasnosti moderným frameworkom.
      \item Backend je implementovaný jednoduchým frameworkom, ktorý
            HTTP dotazy smeruje na jednotlivé funkcie.
      \item Používateľské rozhranie aplikácie je responzívne pre desktopové a mobilne zariadenia.
      \item Vizuál stránky, tým je myslené ako stránka vyzerá, je jednoduchý z dôvodu, že hlavný zámer stránky
            je funkcionalita a nie vizuálny vzhľad.
      \item Aplikácia využíva Wikidata Query service pre získavanie všetkých potrebných dát.
      \item Aplikácia je určite kompatibilná s prehľadávačom Google Chrome.
      \item Dáta si aplikácia ukladá do databázy, ktorá je vhodná a jednoduchá na implementovanie.
      \item Zdrojový kód aplikácie je prehľadný. Objekty, metódy a funkcie sú zdokumentovane.
      \item Komponenty aplikácie sú navrhnute tak aby sa dali recyklovať a aby nedokázalo k duplikovaniu kódu.
\end{enumerate}


\chapter{Mapové webové aplikácie}
V tejto kapitole si prejdeme konkrétne vybrané webové aplikácie, ktoré pracujú s mapovými dátami a poskytujú používateľom určitú funkcionalitu. Pre každú vybranú webovú aplikáciu popíšeme 
funkcie z pohľadu ponúknutia rôznych druhov máp, vyhľadávania objektu a skupiny objektov na základe zadaných parametrov, 
poskytovania informácii o objekte a spravovania mapových objektov. 
Spomínane funkcionality si popíšeme, následne porovnáme a uvedieme, v akých funkciách sa bude naša aplikácia odlišovať od ostatných analyzovaných 
aplikácii. Pojmom odlišovať myslime aké funkcie budú v našej aplikácii chýbať , ktoré budú upravene a na druhu 
stranu aké funkcie budú rozširovať funkcionalitu našej aplikácie. 

Webových aplikácii zaoberajúcich sa mapovými dátami existuje veľké množstvo.
Z tohoto dôvodu vyberieme iba zopár tých najznámejších na zakladá našich vlastných 
poznatkov a skúsenosti s nimi. 
Vybrane webové mapové aplikácie sú Google Maps, Mapy.cz a OpenStreetMap. 

\section{Google Maps}
Google Maps je online webová aplikácia zaoberajúca sa prácou s mapami. Túto webovú aplikáciu vlastní a poskytuje verejnosti spoločnosť Google. 
Podľa metriky k roku 2020 používa mesačne Google Maps služby vyše než jedna miliarda aktívnych užívateľov.(1) 
Aplikácia ponúka rožné druhy máp ako satelitnú mapa, Street view, mapu zobrazujúca cestnú premávku, terénnu mapa a tak ďalej.
Naša aplikácia bude poskytovať používateľovi iba jeden druh mapy a to klasicky. 

Aplikácia ponúka používateľovi vyhľadávať mapové objekty na základe zadania kľúčovej hodnoty názvu objektu. Zároveň vyhľadávanie podporuje možnosť zadať názov kategórie, ako napríklad “hrady” alebo “múzea”, a následne vyhľadávať skupinu objektov, ktorá spadá do tejto kategórie. 
Vyhľadávanie skupiny objektov podľa kategórie je obmedzené iba na vyhľadávanie objektov v okolí konkrétneho miesta. Respektíve v obdĺžnikovej oblasti, ktorá sa používateľovi zobrazuje na mape. 
Ak má používateľ otvorenú mapu, ktorá zobrazuje na mape  mesto Praha a používateľ uvedie do vyhľadávania kľuč "múzea", 
Aplikácia vyhľadá múzea iba v meste Praha. Keď používateľ ma dostatočne oddialenú mapu tak aby mu mapa ukazovala cely svet a následne uvedie do vyhľadávania názov kategórie a potvrdí vyhľadávanie, 
aplikácia nájde nejaké výskyty objektov v danej oblasti. Nájdená skupina objektov je samozrejme neúplná. To znamená, že neobsahuje všetky objekty v hľadanej oblasti. Kým na malom území vyhľadávanie nájde všetky objekty patriace ku hľadanej kategórii, tak čim väčšia oblasť tak tým sú výskoty viacej rozptýlene, teda vyhľadávanie nájde iba niektoré objekty.  
Od našej aplikácie očakávame ,že umožni vyhľadávať všetky objekty na území, ktoré používateľ definuje. Zatiaľ čo Google Maps vyhľadávajú iba v zobrazenej oblasti, tak naša aplikácia umožni si vybrať a definovať spôsob vymedzenia oblasti vyhľadávania.  
Jedná sa o možnosť vybrať a definovať územie ako administratívnu oblasť, krajinu, región alebo vyhľadávať objekty na celom svete. 

Vyhľadávanie podľa kategórie nepodoprú definovať výnimky vo vyhľadávaní. Napríklad pre hodnotu "hrad" nám aplikácia nájde hrady, ale aj objekty zrúcanín hradov. 
Ak používateľ zadá ako kľuč "zrúcanina hradu" tak mu aplikácia vyhľadá iba zrúcaniny hradov. Možnosť vyhľadať skupinu objektov reprezentujúcu hrady ale vylúčiť z nej objekty zrúcanín hradov  Google Maps nepodporujú. 
Od našej aplikácie túto voliteľnú možnosť budeme očakávať. Inými slovami používateľ bude schopný pre vybranú kategóriu zvoliť pod-kategóriu, ktorá obmedzí vyhľadávanie. 

Vyhľadávanie taktiež neumožni používateľovi definovať a zadať nejaké vlastnosti hľadaných objektov, ktoré by nájdene objekty museli splňovať. Uveďme ako príklady zopár rozumných vlastnosti. 
Napríklad vlastnosť “nadmorská výška”, kde používateľ uvedie hodnotu v niektorej z rozumných jednotkách a vyberie operátor porovnávania. Nájdene objekty musia na základe operátora porovnania spĺňať nadmorskú výšku. Napríklad musia byť položene vo viac ako 500 metrov nad morom. 
Vlastnosť “architekt budovy”, kde používateľ potrebuje  
vyhľadať architekta a vyhľadávanie nájde všetky budovy, ktoré uvedený architekt navrhol. Posledná uvedená vlastnosť ako príklad je “čas vzniku”, kde by používateľ uviedol časový udaj ako napríklad storočie a aplikácia by mu vyhľadala objekty, ktoré vznikli respektíve boli postavené v tomto storočí. Týchto vlastnosti existuje obrovské množstvo a od našej aplikácie očakávame,
že umožni vyhľadávať aj s uvedenými obmedzeniami definovanými ako spomínané  vlastnosti. 

Google Maps poskytujú používateľovi rožne informácie o objekte na základe, kategórie čo daný objekt reprezentuje. 
Napríklad pre objekty, ktoré sú mesta aplikácia poskytne informácie ako lokálny čas v danom meste a 
aktuálne počasie. 
Pre objekty reštaurácie poskytne informácie ako hodnotenie reštaurácie a recenzie na danú reštauráciu od používateľov. 
Ako posledný príklad uvedieme hotely, kde aplikácia získa informáciu o cenne hotela za noc a ponúkne odkazy na stránky, 
kde používateľ je schopný si objednať noc v hoteli. 
Takéto informácie podľa kategórie od našej aplikácie nebudeme očakávať, z dôvodu potreby získavania informácii z rôznych iných služieb. 
Naša aplikácia bude využívať iba informácie, ktoré poskytujúcu Wikidata. 

Pre všetky objekty služba poskytne obrázok objektu, a základne informácie ako adresu, kde sa konkrétny objekt nachádza, 
webovú stránku objektu, telefónne číslo, otváracie hodiny a recenzie používateľov. Uvedene informácie sú statické. Ine informácie ako nadmorská výška a podobne služba používateľovi neposkytne. 
Naša aplikácia síce nebude podporovať poskytovanie používateľských recenzii, ale pokúsime sa aby poskytovala rôzne informácie, ktoré Wikidata ponúkajú. 
Napríklad pre mestá nám naša aplikácia získa aj údaj o počte obyvateľstva. 

Mapové objekty Google maps umožnia používateľovi si pridať objekt do listu, kde bude objekt uložený. Pre každý uložený objekt je používateľ schopný 
napísať poznámky k objektu, ktoré vie vždy editovať. Google Maps poskytujú možnosť zaznamenávania histórie návštevnosti. To funguje nasledovným spôsobom. Používateľ 
povolí aby Google Maps mohli používať jeho GPS dáta a z toho automaticky dôkaze aplikácia usúdiť, že používateľ navštívil konkrétny objekt. 
Od naša aplikácia očakávame podobnú možnosť zaznamenávania návštevnosti, ale bez automatického zaznamenávania návštevy objektov. Používateľ našej 
aplikácie bude musieť návštevnosť manuálne nastaviť. Naša aplikácia bude zároveň podporovať to, že ikonky na mape reprezentujúce objekty, ktoré 
ešte neboli navštívene budú bez farby, teda iba čisto sivé, a navštívene budú obsahovať plnú farbu ikonky obrázka. To používateľovi umožni odlišovať navštívene objekty od zatiaľ nenavštívených.  

Google Maps poskytujú aj možnosť vytvorenia a editovania vlastnej mapy, kde používateľ je schopný uložiť mapové body. 
Týmto bodom je schopný nastaviť farbu a obrázok ikonky z ponúknutého výberu, napísať poznámky k objektu a pridať k objektu obrázok alebo video. 
Naša aplikácia z tohto bude podporovať vyber obrázka ikonky ale bez farby, pretože farba reprezentuje v našej aplikácii, či bol konkrétny objekt navštívený.
Zároveň aplikácia umožni požívateľovi vytvoriť poznámky pre objekt. 

Google Maps je webová aplikácia, ktorá umožňuje široké spektrum funkcionalít a poskytovanie zaujímavých informácii o objekte. V tomto aspekte bude naša aplikácia o niečo horšia. 
Na druhu stranu našu aplikáciu obohatíme o precíznejšie vyhľadávanie objektov na základe 
zadaných parametrov ako definovanie kategórie a pod-kategórie objektu, územie a vlastnosti, ktoré musí nájdený objekt spĺňať. 


https://sites.google.com/a/pressatgoogle.com/google-maps-for-iphone/google-maps-metrics


\section{Mapy.cz}
Mapy.cz je webová aplikácia od českej spoločnosti Seznam.cz, ktorá ponuka nielen mapu českej republiky,
ale aj mapu celého sveta vďaka spolupráci s OpenStreetMap. 

Aplikácia ponúka používateľovi na vyber rôzne náhľady máp ako základnú, dopravnú s dopravnými informáciami,
turistickú, leteckú, inak povedané satelitnú a ešte zopár  ako na ako napríklad fotografickú.
Fotografická mapa zobrazuje na mape maličké fotky objektov na miestach kde sa konkrétny objekt nachádza. 
Aplikácia ponuka netradičný náhľad ako na mapu z 19. storočia. Tento náhľad však funguje iba pre českú republiku.
Ako už bolo spomenuté naša aplikácia bude používať iba jeden náhľad a to základný. 

Používateľ je schopný na Mapy.cz vyhľadávať objekty na základe zadania názvu alebo časti názvu objektu ale aj 
na základe zadania názvu kategórie podobne ako u aplikácii Google Maps. 


Mapy.cz nájdu skupinu objektov v obdĺžnikovej oblasti ktorá sa pravé zobrazuje používateľovi na mape. 
Rovnako ako u Goole Maps čim je zobrazená oblasť väčšia je tým nájdene objekty sú viac rozptýlene. Teda aplikácia nájde iba niektoré 
objekty z každej časti mapy. Aplikácia nenájde všetky objekty patriace do kategórie. Keď si používateľ priblíži 
na menšiu oblasť a znovu zadá vyhľadávanie, tak aplikácia nájde viac objektov v danej oblasti. 
Aby aplikácia používateľovu vyhľadala všetky objekty v danej kategórii musí sa vyhľadávať na malom území.  
Mapy.cz neumožňujú vyhľadávať v oblasti ktorú si definuje používateľ. Jedná sa o rovnaký problém ako u Google Maps,
ktorý v našej aplikácii budeme riešiť možnosťou definovať oblasť vyhľadávania. 

Zároveň ani Mapy.cz neposkytujú možnosť definovať pod-kategóriu, ktorá obmedzuje vyhľadávanie aby sa objekty tejto pod-kategórii nevyhľadávali. 

Možnosť obmedziť vyhľadávanie skupiny mapových objektov, tým že použivateľ definuje vlastnosti,
ktoré musia nájdene objekty, 
spĺňať podobne ako Google Maps aj aplikácia Mapy.cz tuto možnosť nepodporuje. 

Pre vyhľadané alebo zvolené objekty Mapy.cz poskytujú používateľovi informácie ako obrázok objektu, popis objektu získaní z Wikipédií a informáciu o 
počasí podobne ako Google Maps, ale oproti Google Maps Mapy.cz poskytujú aj detailnejšie informácie na základe toho čo sa dá o danom objekte povedať. 
Ako príklad si uvedieme, že pre objekt kategórii mesto Mapy.cz poskytli informáciu o populácii a nadmorskej výške. Tieto informácie Google Maps neposkytujú pre mesto.

Týchto informácii o objekte nebolo obrovské množstvovo a používateľ v nich nenájde všetky možne informácie.
Od našej aplikácie budeme očakávať, že pre objekt získa viac informácii z Wikidata.
A to v podobe vypísania vlastnosti objektu a uvedenia ich hodnôt. Pre objekt nebudeme staticky definovať,
ktoré vlastnosti to budú. Aplikácia poskytne všetky rozumne vlastnosti, ktoré sa budú dať získať z Wikidata. 

Mapy.cz taktiež poskytujú možnosť si ukladať a spravovať mapové objekty. 

\section{OpenStreetMap}
OpenStreetMap poskytujú mapové dáta pre webové a mobilné aplikácie. Dáta na OpenStreetMap sú otvorene. 
To znamená, že sa môžu verejne šíriť a upravovať. 

Aj služba OpenStreetMap ponúka používateľovi viaceré druhy máp.  
Sú to štandardná, cyklistická mapa, ktorá poskytuje a zobrazuje používateľovi dáta o cyklistických 
cestách formou farebného vyznačenia ciest, dopravná alebo humanitárna mapa. 
Nevýhodou pre používateľa oproti spomínaním aplikáciám je absencia satelitnej mapy. 

OpenStreetMap poskytujú zaujímavú funkciu “Prieskum prvkov”. Tato funkcia vyhľadá okolité
objekty okolo bodu, na ktorý používateľ klikne na mape. Táto oblasť je ale veľmi mala.
Vyhľadávanie nájde rôzne objekty ako aj cesty, chodníky alebo dokonca stromy. 
Aplikácia poskytuje ku každému objektu  informácie , ktoré opisujú a charakterizujú konkrétnejšie objekt.

Vyhľadávanie objektu alebo objektov je na OpenStreetMap odlišne od Mapy.cz a Google Maps. 
Nájdene výsledky sa nezobrazia rovno na mape. Zobrazia sa iba ako zoznam objektov. 
Bod na mape sa objaví iba jeden, ktorý reprezentuje ten objekt, na ktorý používateľ 
klikol v zozname nájdených objektov. Pre každý nájdený  objekt aplikácia poskytne názov 
kategórie, pod ktorú objekt najviac spadá. 

Aplikácia neumožňuje používateľovi vyhľadávať objekty podľa kategórie.
Táto funkcionalita tu značne chyba oproti spomínaním dvom aplikáciám. 

Na rozdiel od Google Maps a Mapy.cz sa OpenStreetMap líšia hlavné  neposkytovaním funkcionality na 
manažovanie objektov. OpenStreetMap sa zameriava skôr na poskytovanie mapových dát. 
V OpenStreetMap si používateľ nevie uložiť objekt do listu a následne k objektu si napísať poznámky ,alebo 
pridať fotku. 

OpenStreetMap sú v tejto trojici najmenej používateľsky vhodné a neposkytujú
takú funkcionalitu ako zvyšné dve aplikácie.
Najsilnejšou stránkou je ale možnosť voľného používania a  šírenia dát.
Naša aplikácia podobne ako Mapy.cz bude OpenStreetMap využívať a to v podobe 
používania štandardnej mapy. 


